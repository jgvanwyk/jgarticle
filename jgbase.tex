% Load the article class, forwarding all provided options to it
\DeclareOption*{\PassOptionsToClass{\CurrentOption}{article}}
\ProcessOptions\relax
\LoadClass{article}

% Packages

\RequirePackage{xparse}
\RequirePackage{xcolor}
\RequirePackage{amsmath}
\RequirePackage{mathdots}
\RequirePackage{hyperref}
\RequirePackage{amsthm}
\RequirePackage{mathtools}
\RequirePackage{newtxtext,newtxmath}
\RequirePackage[cal=boondox]{mathalpha}
\RequirePackage{enumitem}
\RequirePackage{tikz-cd}
\RequirePackage[nameinlink,capitalise]{cleveref}

% Styling
\setlist{%
    label         = (\alph*),
    listparindent = \parindent,
    topsep        = \smallskipamount,
    parsep        = \parskip,
    left          = .5ex}

\tikzcdset{arrow style = math font}

\if@titlepage
    \let\articlemaketitle\maketitle
    \renewcommand{\maketitle}{\hypersetup{pageanchor=false}\articlemaketitle}
\fi

% Spacing around display environments
\AtBeginDocument{%
    \bigskipamount.7\baselineskip plus.7\baselineskip
      \medskipamount\bigskipamount \divide\medskipamount\tw@
      \smallskipamount\medskipamount \divide\smallskipamount\tw@
      \abovedisplayskip\medskipamount
      \belowdisplayskip \abovedisplayskip
      \abovedisplayshortskip\abovedisplayskip
      \advance\abovedisplayshortskip-1\abovedisplayskip
      \belowdisplayshortskip\abovedisplayshortskip
      \advance\belowdisplayshortskip 1\smallskipamount
      \jot\baselineskip \divide\jot 4 \relax}

\newcommand{\defn}{\emph}
\newcommand{\comment}[1]{\textcolor{red}{#1}}

\newcommand{\deq}{\vcentcolon=}
\newcommand{\deqby}{=\vcentcolon}
\newcommand{\iso}{\simeq}
\newcommand{\comp}{\circ}

% Overrides
\renewcommand{\Gamma}{\varGamma}
\renewcommand{\Lambda}{\varLambda}
\renewcommand{\Omega}{\varOmega}

\renewcommand{\phi}{\varphi}
\renewcommand{\epsilon}{\varepsilon}
\renewcommand{\emptyset}{\varnothing}

\let\Re\@jgundefined
\let\Im\@jgundefined
\DeclareMathOperator{\Re}{Re}
\DeclareMathOperator{\Im}{Im}

\renewcommand{\implies}{\Rightarrow}
\renewcommand{\iff}{\Leftrightarrow}

\DeclareMathOperator{\id}{id}
\DeclareMathOperator{\Id}{Id}
\DeclareMathOperator{\supp}{supp}
\DeclareMathOperator{\dist}{dist}
\DeclareMathOperator{\im}{im}
\DeclareMathOperator{\coker}{coker}
\DeclareMathOperator{\sgn}{sgn}
\DeclareMathOperator*{\Res}{Res}

% Delimiters
\DeclarePairedDelimiter{\paren}{(}{)}
\DeclarePairedDelimiter{\abs}{\lvert}{\rvert}
\DeclarePairedDelimiter{\modl}{\lvert}{\rvert}
\DeclarePairedDelimiter{\norm}{\lVert}{\rVert}
\DeclarePairedDelimiter{\floor}{\lfloor}{\rfloor}
\DeclarePairedDelimiter{\ceil}{\lceil}{\rceil}
\DeclarePairedDelimiter{\innprod}{\langle}{\rangle}

% Sets
\let\numset\vvmathbb
\newcommand{\RR}{\numset{R}}
\newcommand{\CC}{\numset{C}}
\newcommand{\QQ}{\numset{Q}}
\newcommand{\NN}{\numset{N}}
\newcommand{\KK}{\numset{K}}
\newcommand{\ZZ}{\numset{Z}}
\newcommand{\FF}{\numset{F}}
\newcommand{\Sph}{\numset{S}}
\newcommand{\Trs}{\numset{T}}
\newcommand{\CP}{\numset{CP}}
\newcommand{\RP}{\numset{RP}}
\newcommand{\SymGrp}{\mathfrak{S}}

\providecommand{\st}{}
\DeclarePairedDelimiterX{\set}[1]{\{}{\}}{%
    \renewcommand{\st}{:}%
    #1}
\DeclarePairedDelimiterX{\grp}[1]{\langle}{\rangle}{%
    \renewcommand{\st}{:}%
    #1}

% Modifiers
\renewcommand{\vec}{\mathbf}
\newcommand{\utilde}[1]{\underaccent{\tilde}{#1}}

% Linear algebra
\newcommand{\dsum}{\oplus}
\newcommand{\dprod}{\times}
\newcommand{\tprod}{\otimes}
\newcommand{\kprod}{\otimes}
\newcommand{\bigdprod}{\prod}
\newcommand{\bigdsum}{\bigoplus}
\newcommand{\bigtprod}{\bigotimes}
\newcommand{\inclus}{\hookrightarrow}

\DeclareMathOperator{\Mat}{Mat}
\DeclareMathOperator{\GL}{GL}
\DeclareMathOperator{\Hom}{Hom}
\DeclareMathOperator{\End}{End}
\DeclareMathOperator{\Sp}{Sp}
\DeclareMathOperator{\SO}{SO}
\newcommand{\Ext}{\Lambda}
\newcommand{\Sym}{S}

\newcommand{\pmat}[1]{\begin{pmatrix}#1\end{pmatrix}}
\newcommand{\bmat}[1]{\begin{bmatrix}#1\end{bmatrix}}
\newcommand{\Bmat}[1]{\begin{Bmatrix}#1\end{Bmatrix}}
\newcommand{\vmat}[1]{\begin{vmatrix}#1\end{vmatrix}}
\newcommand{\Vmat}[1]{\begin{Vmatrix}#1\end{Vmatrix}}

\newcommand{\blank}{{-}}

% Fourier transform
\newcommand{\Sw}{\mathcal{S}}
\newcommand{\Ft}{\mathcal{F}}
\newcommand{\conv}{\ast}

% Manifolds
\newcommand{\Sec}{\Gamma}
\newcommand{\conn}{\nabla}
\newcommand{\lapl}{\Delta}
\newcommand{\hstar}{\ast}
\DeclareMathOperator{\divg}{div}
\newcommand{\hook}{\mathop{\lrcorner}}
\newcommand{\del}{\partial}
\newcommand{\DForm}{\Omega}

% Functional analysis
\newcommand{\BLOP}{\mathcal{B}}

% Derivatives
\NewDocumentCommand{\dv}{smo}{\IfBooleanTF{#1}%
    {\frac{d\IfValueT{#3}{#3}}{d#2}}%
    {d\IfValueT{#3}{#3}/d#2}}

\NewDocumentCommand{\mdv}{smmo}{\IfBooleanTF{#1}%
    {\frac{d^{#3}\IfValueT{#4}{#4}}{d#2^{#3}}}%
    {d^{#3}\IfValueT{#4}{#4}/d#1^{#3}}}

\NewDocumentCommand{\pdv}{smo}{\IfBooleanTF{#1}%
    {\frac{\partial\IfValueT{#3}{#3}}{\partial #2}}%
    {\partial\IfValueT{#3}{#3}/\partial #2}}

\NewDocumentCommand{\@PdvParseVariable}{>{\SplitList{,}}m}{\ProcessList{#1}{\@PdvParseExponent}}
\NewDocumentCommand{\@PdvParseExponent}{m}{\partial #1}

\NewDocumentCommand{\mpdv}{s>{\SplitList{,}}mmo}{\IfBooleanTF{#1}%
    {\frac%
        {\partial^{#3}\IfValueT{#4}{#4}}
        {\ProcessList{#2}{\@PdvParseVariable}}}%
    {\partial^{#3}\IfValueT{#4}{#4}/\ProcessList{#2}{\@PdvParseVariable}}}

% Meta
\newcommand{\omitted}[1]{\widehat{#1}}

\NewDocumentCommand{\evalat}{smm}{\IfBooleanTF{#1}%
    {\mathopen{}%
        \kern-\nulldelimiterspace%
        \left.%
        #2%
        \right\rvert_{#3}%
        \mathclose{}}%
    {#2\rvert_{#3}}}

\let\restto\evalat
